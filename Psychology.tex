% !TEX program = xelatex
% !Mode:: "TeX:UTF-8"

% LaTeX template for Chinese essays in psychology
% Author: Yumeng Jiang, Dept. of Psychology, PKU

% Please typeset with xelatex + bibtex + xelatex + xelatex

\documentclass[UTF8,a4paper]{ctexart}

% ==========Preamble==========

\usepackage{apacite}
\usepackage{fancyhdr}
\usepackage{geometry}
\usepackage[font=small,labelfont=bf,labelsep=quad,format=hang,textfont=it]{caption}
\usepackage{booktabs}
\usepackage{graphicx}
\usepackage{float}

\pagestyle{plain}
\CTEXsetup[format=\Large\bfseries]{section}
\bibliographystyle{apacite}

% ==========Title==========

\title{\bfseries 工作记忆与挖掘机学习} 
\author{荣兰祥 \thanks{山东蓝翔职业技术学校校长;}} 
% Your name in the first blank and your additional information in \thanks{}
\date{\today}
% delete \today if you don't want the date

% ==========Document==========

\begin{document}

\maketitle

% ==========Abstract==========

\begin{center}
\parbox{130mm}{\zihao{-5}{\bfseries 摘\quad 要:}
% Abstract here
% An example is as follows
本研究主要研究了工作记忆和挖掘机学习之间的影响,通过……发现……
\par
\vspace{1mm}
{\bfseries 关键词:}工作记忆\quad 挖掘机\quad 蓝翔技校}
\end{center}

% ==========Body==========

% Example article

\section{前言}
通过对前人的研究的归纳总结,我们发现工作记忆对学习有非常重要的影响现在问题来了,挖掘机技术哪家强?工作记忆对挖掘机技术有什么影响?这就是我们这项研究要解决的问题。

\section{方法}
\subsection{被试}
本研究从山东蓝翔职业技术学校和五道口职业技术学校各招募了20名被试,每个学校男女各一半,……

\subsection{仪器和刺激}
我们有fMRI有木有!
刺激就是一堆麻麻的点有木有!

\subsection{实验设计}
我们是三因素混合设计,其中有一个组内变量(上楼梯、下楼梯),组间变量分别是性别和挖掘机类型。

\subsection{实验流程}
好复杂啊,我可以懒的细写了,宝贝你就好好写吧~

\subsection{结果}
首先是描述性统计,大概感觉一下哈~
然后好多验证性统计啊,$t$检验就不做了,做ANOVA会好多。等我学会,给你做结构方程啊多因素回归让你看不懂。
对了,你看看我做的图是不是很好看?

\section{讨论}
\citeA{baddeley1992working}对工作记忆有非常系统的论述。通过这个研究我们发现,挖掘机的学习和工作记忆有着非常强的关联。……

% ========About apacite===========
% \cite{baddeley1992working}  => (Baddeley, 1992)
% \citeA{baddeley1992working} => Baddeley (1992)

% ==========References==========
\renewcommand{\refname}{参考文献}
% enter your .bib file name in the parentheses in the following line.
\bibliography{ref}

\clearpage
\section{附录}
挖掘机参数评价表


\end{document}
